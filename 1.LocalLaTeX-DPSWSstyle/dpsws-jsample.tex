\documentclass[submit,techreq,noauthor]{dpsws}

\usepackage[dvips]{graphicx}
\usepackage{latexsym}

\def\Underline{\setbox0\hbox\bgroup\let\\\endUnderline}
\def\endUnderline{\vphantom{y}\egroup\smash{\underline{\box0}}\\}
\def\|{\verb|}

\begin{document}


\title{DPSWS原稿の準備方法(2022年6月15日版)}

\affiliate{IPSJ}{情報処理学会\\
IPSJ, Chiyoda, Tokyo 101--0062, Japan}

\paffiliate{JU}{情報処理大学\\
Johoshori Uniersity}

\author{情報 太郎}{Joho Taro}{IPSJ}[joho.taro@ipsj.or.jp]
\author{処理 花子}{Shori Hanako}{IPSJ}
\author{学会 次郎}{Gakkai Jiro}{IPSJ,JU}[gakkai.jiro@ipsj.or.jp]

\begin{abstract}
本稿は,マルチメディア通信と分散処理ワークショップ(DPSWS)に投稿する
原稿を執筆する際のフォーマット及び注意点をまとめたものである.
本稿も投稿フォーマットに従って執筆されているため,
著者は本稿のソースファイルを雛形にして原稿を執筆することが可能である.
DPSWSで用いるスタイルファイルは,情報処理学会論文誌のスタイルファイル
(http://www.ipsj.or.jp/journal/submit/style.html からアクセス可能)
を継承しているため,使用すべき{\LaTeX}コマンドや執筆形式の詳細については,
そちらをご参照いただきたい.
\end{abstract}

\maketitle

%1
\section{投稿フォーマットについて}

マルチメディア通信と分散処理ワークショップ(DPSWS)に投稿する論文を
{\LaTeX}を利用して作成する場合には、\texttt{dpsws.cls}ファイルを用いることとする.
和文論文の場合には,texソースの冒頭に次のように記述すること.\\

\begin{verbatim}
\documentclass[submit,techreq,noauthor]{dpsws}

\end{verbatim}

\noindent
英文論文の場合には,次のように記述すること.\\

\begin{verbatim}
\documentclass[techreq,english]{dpsws}

\end{verbatim}

クラスファイル\texttt{dpsws.cls}は情報処理学会標準の\texttt{ipsj.cls}を基にして,
ヘッダ,フッタを出力しないようにカスタマイズした
ものであり(和文論文の場合には英語タイトル,英文著者名,
及び英文アブストラクトも出力しないようにしてある),
情報処理学会の許諾の下に配布している.
その他の本論文の体裁については
「情報処理学会論文誌(IPSJ Journal) 原稿執筆案内」
\cite{ipsjFormat}に基づいて記述することとする.
但し,biographyセクションは記述しないものとする.

これらスタイルファイルについて,
情報処理学会に問い合わせることはしないようにお願いしたい.
DPSWSとしてもスタイルファイルに対するサポートは行わないが,
不備や不明な点等があり問い合わせが必要である場合には,
DPSWSの問い合わせ窓口にご連絡いただきたい.


\section{文字コードについて}

DPSWS向けに用意された\texttt{dpsws.cls}や\texttt{dpsws.sty}などの
ファイルは,UTF-8の文字コードで作成している.
他の環境で執筆する場合は,適宜文字コードや改行コードを変換してから利用されたい.

\begin{thebibliography}{10}

\bibitem{ipsjFormat}
情報処理学会論文誌(IPSJ Journal)原稿執筆案内
(\urlj{http://www.ipsj.or.jp/journal/submit/ronbun\_j\_prms.html})
(2022.06.15).

\end{thebibliography}


\end{document}
